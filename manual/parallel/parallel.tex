\documentclass{article}
\usepackage{amssymb,amsfonts,amsmath,amscd,amsthm}
\usepackage{graphicx}
\usepackage{rotating}
\usepackage{color}
\usepackage{bm}
\usepackage{booktabs}
\usepackage{fancyvrb}
\usepackage{booktabs}  %nice tables

\definecolor{dkgreen}{rgb}{0,0.6,0}
\definecolor{gray}{rgb}{0.5,0.5,0.5}
\definecolor{mauve}{rgb}{0.58,0,0.82}

\newcommand{\Quesoweb}{\url{https://github.com/libqueso}}
\newcommand{\Queso}{\texttt{QUESO}}
\newcommand{\QUESOversion}{0.51.0}

\usepackage{listings}          % to include codes using \lstinputlisting
\lstset{
language=c++,                  % choose the language of the code
basicstyle=\ttfamily,          % the size of the fonts that are used for the code
%stringstyle=\ttfamily,
%keywordstyle=\bfseries,       % so funciona com basicstyle=\footnotesize,\ttfamily se eu adicionar \usepackage{bold-extra}
keywordstyle=\color{blue},     % keyword style
commentstyle=\color{dkgreen},  % comment style
stringstyle=\color{mauve},
identifierstyle=\bfseries,
numberbychapter= true,
numberfirstline=false,
% numbers=left,                % where to put the line-numbers
numberstyle=\footnotesize,     % the size of the fonts that are used for the line-numbers
stepnumber=5,                  % the step between two line-numbers. If it is 1 each line will be numbered
numbersep=8pt,                 % how far the line-numbers are from the code
showspaces=false,              % show spaces adding particular underscores
showstringspaces=false,        % underline spaces within strings
showtabs=false,                % show tabs within strings adding particular underscores
tabsize=2,                     % sets default tabsize to 2 spaces
captionpos=b,                  % sets the caption-position to bottom
breaklines=true,               % sets automatic line breaking
breakatwhitespace=false,       % sets if automatic breaks should only happen at whitespace
escapeinside={\%*}{*)},        % if you want to add a comment within your code
morekeywords ={rm,ls},
belowskip = 10pt,              % \medskipamount%\smallskipamount,
aboveskip =10pt,
}

\title{Distributed parameter states in \Queso}
\author{Damon McDougall}

\begin{document}

\maketitle

\section{Introduction}

This document describes the current, serial, state of the parameter vector in
\Queso\ and makes a plan to transition to a distributed state.

\subsection{The current state}

The current state of the parameter vector in \Queso\ is serial.  Parallelism in
\Queso\ presents itself in two ways: 1) independent parallel Markov chains that
execute concurrently; and 2) a mechanism, an MPI communicator, that \Queso\
creates as part of the construction of \lstinline|FullEnvironment| the user can
hand to a forward problem demanding parallelism.  Neither of these parallel
capabilities distribute the parameter vector across multiple processes.  In
other words, each chain's process holds exactly the same parameter vector
value in the likelihood.

Here is some example commented code that executes the current situation:
\begin{lstlisting}
#include <queso/Environment.h>
#include <queso/GslVector.h>
#include <queso/GslMatrix.h>
#include <queso/ScalarFunction.h>
#include <queso/VectorSpace.h>
#include <queso/BoxSubset.h>

using namespace QUESO;

template <class V = GslVector, class M = GslMatrix>
class Likelihood : public BaseScalarFunction<V, M>
{
public:
  Likelihood(const char * prefix,
             const VectorSet<V, M> & domainSet)
    : BaseScalarFunction<V, M>(prefix, domainSet),
      m_env(domainSet.env())
  {
  }

  virtual double lnValue(const V & param) const
  {
    if (m_env.subRank() == 0) {
      std::cout << "Rank 0 param: " << param << '\n';
    }
    else {  // Rank 1
      std::cout << "Rank 1 param: " << param << '\n';
    }

    return 1.0;
  }

  virtual double actualValue(const V&, const V*, V*,
                             M*, V*) const
  {
    return 1.0;
  }

  const BaseEnvironment & m_env;
};

int main(int argc, char ** argv)
{
  // We'll assume the program was executed with
  // mpirun -np 2 and there's only one chain.

  MPI_Init(&argc, &argv);
  FullEnvironment env(MPI_COMM_WORLD,
                      argv[1],
                      "",
                      NULL);
  VectorSpace<> paramSpace(env, "", 1, NULL);
  GslVector min(paramSpace.zeroVector());
  GslVector max(paramSpace.zeroVector());
  min[0] = 0.0;
  max[0] = 1.0;
  BoxSubset<> paramDomain("", paramSpace, min, max);
  Likelihood<> likelihood("", paramDomain);

  GslVector point(paramSpace.zeroVector());
  point[0] = 0.5;
  likelihood.lnValue(point);  // Both ranks should
                              // print the same
                              // parameter value

  MPI_Finalize();
  return 0;
}
\end{lstlisting}

\subsection{Problems with the current state}

\subsection{Why we need a distributed state}

computation, licencing

\section{How the future might look}

\subsection{From \Queso's perspective}
\subsection{From the user's perspective}

\end{document}
